Der foretages en MoSCoW analyse for at få et overblik over Applikationens ønskede funktionalitet, samt at prioritere implementeringen af applikationens forskellige elementer. MoSCoW analyse danner således baggrund for udviklingsprocessen.
\newline\newline

\noindent \textbf{Must have:}

\begin{itemize}
    \item Applikationen skal have en login-funktion og mulighed for at oprette brugere
    \item En bruger skal identificeres ved e-mail som anvendes ved login
    \item I applikationen skal det være muligt at oprette en gruppe med et default layout, hvortil der kan inviteres medlemmer
    \item Applikationen skal indeholde en kalender
    \item Applikationen skal indeholde funktionalitet der gør det muligt at se og planlægge en madplan og rengøringsplan
    \item Det skal være muligt at bytte rengørings- og mad- vagter mellem brugere. 
    \item Gruppens aktiviteter skal vises på gruppens kalender
    \item Applikationen skal indeholde en opslagstavle for en gruppe.
    \item Der skal kunne oprettes forbindelse til serveren ved brug af HTTP
    \item Applikationen skal indeholde en liste widget.
    \item Det skal være muligt for gruppens medlemmer at tilføje og slette forskellige widgets på gruppens side.
    \item Applikationen skal gøre det muligt at oprette et booking system til booking af ressourcer
    \item Applikationen skal have en regnskabs widget.
\end{itemize}

\noindent \textbf{Should have:}

\begin{itemize}
    \item Mulighed for at tilpasse udseendet af gruppen
    \item Administrator rettigheder
\end{itemize}

\noindent \textbf{Could have:}

\begin{itemize}
    \item Mulighed for at indsamle brugerstatistikker
    \item Integrering med dansk supermarked der udbyder levering af varer
\end{itemize}

\noindent \textbf{Won't have (this time)}
\begin{itemize}
    \item En tilsvarende App til iOS/Android
    \item Integrering med mobilePay/ weshare mm.
\end{itemize}

