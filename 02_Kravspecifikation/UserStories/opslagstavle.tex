\subsubsection{Opslagstavle}


% \subsubsection{Tilf{\o}j opslag}
% \textbf{Som} bruger\\
% \textbf{Vil jeg} have mulighed for at tilføje opslag til opslagstavlen,\\
% \textbf{For at} jeg kan lave opslag som andre i gruppen kan læse.

% \subsubsection{Fjern opslag}
% \textbf{Som} administrator\\
% \textbf{Vil jeg} have mulighed for at fjerne opslag fra opslagstavlen, \\
% \textbf{For at} jeg kan fjerne irrelevante/stødende opslag.

% \subsubsection{Tilføj brugere}
% \textbf{Som} administrator\\
% \textbf{Vil jeg} have mulighed for at tilføje brugere som ikke nødvendigvis er medlem af den aktuelle gruppe,\\
% \textbf{For at} kunne lave en opslagstavle på tværs af grupper.

For at dele opslag i en gruppe kan der oprettes en opslagstavle, hvori gruppens medlemmer kan se og tilføje opslag. Det er muligt at dele opslagstavler på tværs af grupper.\newline

\begin{tabular}{p{2.5in}p{2.5in}}
\textbf{\underline{Tilføj opslag}}\newline \textbf{Som} bruger\newline \textbf{Vil jeg} have mulighed for at tilføje opslag til opslagstavlen,\newline \textbf{For at} jeg kan lave opslag som andre i gruppen kan læse. & 

\textbf{\underline{Fjern opslag}}\newline \textbf{Som} administrator\newline \textbf{Vil jeg}  have mulighed for at fjerne opslag fra opslagstavlen,\newline \textbf{For at} holde opslagene relevante.  \\\\

\textbf{\underline{Tilføj/Fjern gruppe}}\newline \textbf{Som} bruger\newline \textbf{Vil jeg} have mulighed for at tilføje andre grupper
\newline \textbf{For at} kunne lave en opslagstavle på tværs af grupper.  \\\\
\end{tabular}

