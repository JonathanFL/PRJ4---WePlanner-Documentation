\documentclass[11pt,Danish,a4paper,oneside,openright,final]{memoir}

%% Memoir are using an emulation of the ccaption package
%% so to use the caption package without warnings we first
%% DisEmulate the ccaption package after it have been loaded
%% and then call the package we want to use.
%% For more information, check the documentation
%% https://www.ctan.org/pkg/memoir?lang=en
\RequireAtEndPackage{ccaption}{\DisemulatePackage{ccaption}}
\RequireAtEndPackage{ccaption}{\usepackage{caption}}
%\usepackage[danish]{babel}
\usepackage[T1]{fontenc}			% inkluder dansk i overskrifter
\usepackage[utf8]{inputenc}	% inkluder dansk i overskrifter
\usepackage[table,xcdraw,pdftex,dvipsnames]{xcolor}

%% Memoir commands for changeing whitespace before and after 
%% sec, subsec, subsubsec, para or subpara. Replace "S" with 
%% the thing you want to change the whitespace for 
%%(see documentation section 6.6)
%\setbeforeSskip{ <skip> }
%\setafterSskip{ <skip> }

\usepackage[pdftex,dvipsnames]{xcolor}
\usepackage{lipsum}
\usepackage{xargs}

\usepackage[colorinlistoftodos,prependcaption,textsize=tiny]{todonotes}
\newcommandx{\storemathias}[2][1=]{\todo[linecolor=red,backgroundcolor=red!25,bordercolor=red,#1]{#2}}
\newcommandx{\jonathan}[2][1=]{\todo[linecolor=blue,backgroundcolor=blue!25,bordercolor=blue,#1]{#2}}
\newcommandx{\thomas}[2][1=]{\todo[linecolor=green,backgroundcolor=green!25,bordercolor=green,#1]{#2}}
\newcommandx{\tobias}[2][1=]{\todo[linecolor=cyan,backgroundcolor=cyan!25,bordercolor=cyan,#1]{#2}}
\newcommandx{\sejemathias}[2][1=]{\todo[linecolor=violet,backgroundcolor=violet!25,bordercolor=violet,#1]{#2}}
\newcommandx{\florent}[2][1=]{\todo[linecolor=magenta,backgroundcolor=magenta!25,bordercolor=magenta,#1]{#2}}
\newcommandx{\valdemar}[2][1=]{\todo[linecolor=orange,backgroundcolor=orange!25,bordercolor=orange,#1]{#2}}



\usepackage{listings}

\lstset{extendedchars=\true}
\lstset{inputencoding=ansinew}

\lstdefinelanguage{Gherkin}{
  keywords={Naar, Saa, Givet, Og},
  ndkeywords={Egenskab, Baggrund, Scenarie},
  sensitive=false,
  comment=[l]{\#},
  morestring=[b]',
  morestring=[b]"
}

\usepackage{svg}

\lstdefinelanguage{VHDL}{
  morekeywords={
    library,use,all,entity,is,port,in,out,end,architecture,of,
    begin,and
  },
  morecomment=[l]--
}
\usepackage{tabularx}
\usepackage[table,xcdraw]{xcolor}
\usepackage{xcolor}
\colorlet{keyword}{blue!100!black!80}
\colorlet{comment}{green!90!black!90}
\lstdefinestyle{vhdl}{
  language     = VHDL,
  basicstyle   = \ttfamily,
  keywordstyle = \color{keyword}\bfseries,
  commentstyle = \color{comment}
}
\lstdefinestyle{customc}{
  belowcaptionskip=1\baselineskip,
  breaklines=true,
  frame=L,
  xleftmargin=\parindent,
  language=C,
  showstringspaces=false,
  basicstyle=\footnotesize\ttfamily,
  keywordstyle=\bfseries\color{green!40!black},
  commentstyle=\itshape\color{purple!40!black},
  identifierstyle=\color{blue},
  stringstyle=\color{orange},
}
\usepackage{url}
\usepackage{lmodern}
\usepackage{varioref} %% \vref gives you references including pages
\usepackage[fleqn]{amsmath} %
\usepackage[fleqn]{mathtools}				% Andre matematik- og tegnudvidelser
\usepackage[version=3]{mhchem} 				% Kemi-pakke til flot og let notation af formler, f.eks. \ce{Fe2O3}
\usepackage{siunitx}						% Flot og konsistent praesentation af tal og enheder med \si{enhed} og \SI{tal}{enhed}
%\sisetup{locale=DE}							% Opsaetning af \SI (DE for komma som decimalseparator)
\usepackage[utf8]{inputenc}	

\usepackage{pdfpages}			% G�r det muligt at inkludere pdf-dokumenter med kommandoen \includepdf[pages=]{fil.pdf}	
\usepackage{multicol}
\usepackage{multirow}
\usepackage{color} %% Colored text
\usepackage{amsfonts}
\usepackage{amssymb}
\usepackage{amsopn}
\usepackage{latexsym}
\usepackage{amstext}
\usepackage{longtable} %% tables that spans multiple pages
\usepackage{mathrsfs} %% nice math text/symbols

\usepackage{color}
 
\definecolor{dkgreen}{rgb}{0,0.6,0}
\definecolor{gray}{rgb}{0.5,0.5,0.5}
\definecolor{mauve}{rgb}{0.58,0,0.82}
\definecolor{lightgray}{rgb}{0.95,0.95,0.95}

% When including these 3 lines, it now enumerates with C.S.L where C is chapter number, S is section number and L is list number
%\usepackage{enumitem}
%\setenumerate[1]{label=\thesection.\arabic*}
%\setenumerate[2]{label*=\arabic*}
% It is now also possible to continue numbering by using \begin{enumerate}[resume]

\lstset{ %
  language=VHDL,                % the language of the code
  basicstyle=\footnotesize,           % the size of the fonts that are used for the code
  numbers=left,                   % where to put the line-numbers
  numberstyle=\tiny\color{gray},  % the style that is used for the line-numbers
  stepnumber=1,                   % the step between two line-numbers. If it's 1, each line 
                                  % will be numbered
  numbersep=5pt,                  % how far the line-numbers are from the code
  backgroundcolor=\color{lightgray},      % choose the background color. You must add \usepackage{color}
  showspaces=false,               % show spaces adding particular underscores
  showstringspaces=false,         % underline spaces within strings
  showtabs=false,                 % show tabs within strings adding particular underscores
  frame=single,                   % adds a frame around the code
  rulecolor=\color{black},        % if not set, the frame-color may be changed on line-breaks within not-black text (e.g. commens (green here))
  tabsize=2,                      % sets default tabsize to 2 spaces
  captionpos=b,                   % sets the caption-position to bottom
  breaklines=true,                % sets automatic line breaking
  breakatwhitespace=false,        % sets if automatic breaks should only happen at whitespace
  title=\lstname,                   % show the filename of files included with \lstinputlisting;
                                  % also try caption instead of title
  keywordstyle=\color{blue},          % keyword style
  commentstyle=\color{dkgreen},       % comment style
  stringstyle=\color{mauve},         % string literal style
  escapeinside={*(}{)*},            % if you want to add a comment within your code
  morekeywords={*,...},               % if you want to add more keywords to the set
  emph = {STD_LOGIC_VECTOR, STD_ULOGIC_VECTOR, std_logic_vector, std_ulogic_vector, CONV_STD_LOGIC_VECTOR, conv_std_logic_vector,  to_STDULOGICVECTOR, to_stdulogicvector, STD_LOGIC, std_logic} , emphstyle=\color{magenta},
}
%\renewcommand{\lstlistingname}{Kodeeksempel}


\lstset{ %
  language=[Sharp]C,                % the language of the code
  basicstyle=\footnotesize,           % the size of the fonts that are used for the code
  numbers=left,                   % where to put the line-numbers
  numberstyle=\tiny\color{gray},  % the style that is used for the line-numbers
  stepnumber=2,                   % the step between two line-numbers. If it's 1, each line 
                                % will be numbered
  numbersep=5pt,                  % how far the line-numbers are from the code
  backgroundcolor=\color{lightgray},      % choose the background color. You must add \usepackage{color}
  showspaces=false,               % show spaces adding particular underscores
  showstringspaces=false,         % underline spaces within strings
  showtabs=false,                 % show tabs within strings adding particular underscores
  frame=single,                   % adds a frame around the code
  rulecolor=\color{black},        % if not set, the frame-color may be changed on line-breaks within not-black text (e.g. commens (green here))
  tabsize=2,                      % sets default tabsize to 2 spaces
  captionpos=b,                   % sets the caption-position to bottom
  breaklines=true,                % sets automatic line breaking
  breakatwhitespace=false,        % sets if automatic breaks should only happen at whitespace
  title=\lstname,                   % show the filename of files included with \lstinputlisting;
                                  % also try caption instead of title
  keywordstyle=\color{blue},          % keyword style
  commentstyle=\color{dkgreen},       % comment style
  stringstyle=\color{mauve},         % string literal style
  escapeinside={*(}{)*},            % if you want to add a comment within your code
  morekeywords={*,...},               % if you want to add more keywords to the set
  emph = {LOAD , JUMP, COMP, DINT, EINT, STORE, FETCH, RETI, ENABLE} , emphstyle=\color{blue}
}

\makeatletter
\renewcommand*\env@matrix[1][*\c@MaxMatrixCols c]{%
  \hskip -\arraycolsep
  \let\@ifnextchar\new@ifnextchar
  \array{#1}}
\makeatother

%
\setheaderspaces{*}{5\onelineskip}{*}
\makepagestyle{sitin}

% Margin
\setlrmarginsandblock{*}{3.5cm}{0.75} % højre og venstre
\setulmarginsandblock{3cm}{*}{0.75}    % top og bund
\checkandfixthelayout[nearest]        % specifikt valg af højde algoritme
\renewcommand{\marginparwidth}{75pt}

\makeoddhead{sitin}
	%Left
	{
		Gruppe 8
	}
	%Center
	{
		\small\rightmark
	}
	%Right
	{
	    \includegraphics[height=4\onelineskip]{Frontmatter/preamble/LargeAULogo.png}
	}

\makeevenhead{sitin}
	%Left
	{
		\includegraphics[height=4\onelineskip]{Frontmatter/preamble/LargeAULogo.png}
	}
	%Center
	{
		\small\leftmark 
	}
	%Right
	{
	    Gruppe 8
	}
	
\makeoddfoot{sitin}{}{}{\thepage}
\makeevenfoot{sitin}{}{}{\thepage}

\makeheadrule{sitin}{\textwidth}{.4pt}
\makefootrule{sitin}{\textwidth}{.4pt}{0.1cm}

\pagestyle{sitin}



%\usepackage[dvipsnames]{xcolor}
%\newcommand\tab[1][1cm]{\hspace*{#1}}
\usepackage[T1]{fontenc}
\usepackage{afterpage}
\usepackage{siunitx}
\usepackage{transparent}

\usepackage{hyperref}
\usepackage[backend=bibtex,
style=numeric,
bibencoding=ascii
%style=alphabetic
%style=reading
]{biblatex}
%\usepackage[
%singlelinecheck=false % <-- important
%]{caption}

% Change chapter pages
\copypagestyle{chapter}{plain}
\makeoddfoot{chapter}{}{}{\small\thepage}
\makeevenfoot{chapter}{\small\thepage}{}{}
\makefootrule{chapter}{\textwidth}{\normalrulethickness}{\footruleskip}


%
% Section titles
%
\settocdepth{subsection}
\setsecnumdepth{subsection}
\maxsecnumdepth{subsection}
\setsecheadstyle{\Large\bfseries\sffamily\raggedright}
\setsubsecheadstyle{\large\bfseries\sffamily\raggedright}
\setsubsubsecheadstyle{\normalsize\bfseries\sffamily\raggedright}
\raggedbottomsectiontrue

%
% Table of Contents
%
\renewcommand{\contentsname}{Table of Contents}
\makeatletter
\setlength{\cftpartnumwidth}{2em}% Set length of number width in ToC for \part
\setlength{\cftchapternumwidth}{2em}% Set length of number width in ToC for \chapter
\setlength{\cftsectionnumwidth}{3em}% Set length of number width in ToC for \section
\setlength{\cftsubsectionnumwidth}{4em}% Set length of number width in ToC for \subsection
\makeatother


\usepackage{calc}

% Define a new chapter style
\makeatletter
\makechapterstyle{worksheet}{
	%% Memoir commands for changeing whitespace before and after 
	%% chapter. Note that the outcommented values are the defaults
	%% \setlength{\beforechapskip}{50pt}
	%% \setlength{\afterchapskip}{40pt}
	\setlength{\beforechapskip}{1ex}
	\setlength{\midchapskip}{0pt}
	\setlength{\afterchapskip}{1ex}
 	\newcommand{\chapterrule}{\rule[.2\baselineskip]{\textwidth}{1pt}}
  	\renewcommand\chapnamefont{\Large\sffamily}
  	\renewcommand\chapnumfont{\Large\sffamily\centering}
  	\renewcommand\chaptitlefont{\huge\bfseries\sffamily\centering}
  	\renewcommand\printchaptertitle[1]{%
    \chaptitlefont
    	\ifdim\@tempdimc > 0pt\relax% one line
      		\chapterrule \\
      		##1
      		\chapterrule
    	\else% two+ lines
        	>{\chaptitlefont\arraybackslash}p{\textwidth-2\tabcolsep}
     		\chapterrule \\
      		\phantomsection
      		\addtocontents{toc}{\protect\contentsline{chapter}{\protect\numberline{}##1}{}{chapter*.\thepage}}
      		##1
      		\chapterrule
    	\fi
	}
}
\makeatother
\chapterstyle{worksheet}

\usepackage{acronym}
\usepackage{nicefrac}
\usepackage{placeins}
\usepackage{graphicx}
\usepackage{epstopdf} % allows to use eps files in figures
\epstopdfsetup{outdir=./epstopdf/}
\newcommand{\matt}[1]{\bar{\mathbf{#1}}} % Laver Matrix notation med dobbelt overline og fed skrift
\input{Frontmatter/preamble/commands.tex}

% The framed package is used in the example environment
%\usepackage{framed}
% Show the frame of the page segments for placements
%\usepackage{showframe}
% Count chapters, makes it possible to autoupdate number of appendices
\usepackage{totcount}

% To make different kinds of diagrams
\usepackage{tikz}
\usetikzlibrary{calc}
\usepackage{schemabloc} % Documentation only in French 
						% https://www.ctan.org/pkg/schemabloc
\usepackage{blox}		% Does the same as  the schemabloc package, but
						% documentation is in English 
						%https://www.ctan.org/pkg/blox
\usetikzlibrary{circuits}
\usetikzlibrary{positioning}

%%%%%%%%%%%%%%%%%%%%%%%%%%%%%%%%%%%%%%%%%%%
%% IF YOU NEED A PACKAGE, PUT UNDER THIS %%
%%%%%%%%%%%%%%%%%%%%%%%%%%%%%%%%%%%%%%%%%%%

\usepackage[subpreambles=true]{standalone}
\usepackage{import}

\usepackage{tabto}
\usepackage{textcomp}
\usepackage{upgreek}
\usepackage{mathptmx}
\usepackage{array}

\let\newfloat\relax

\newcommand*{\renameenviron}[1]{%
  \expandafter\let\csname exam-#1\expandafter\endcsname
      \csname #1\endcsname
  \expandafter\let\csname endexam-#1\expandafter\endcsname
      \csname end#1\endcsname
  \expandafter\let\csname #1\endcsname\relax
  \expandafter\let\csname end#1\endcsname\relax
}
\renameenviron{framed}
\renameenviron{shaded}
\renameenviron{leftbar}
\renameenviron{snugshade}

\usepackage{circuitikz}%enables electriacls curcits in tikz
%\usepackage{slashbox} % So two different things can be written in a box i an table
\usepackage{diagbox} % modern version of slashbox
\usepackage{cancel} %the ability to strikeout in equations

\usepackage{arydshln} % Dashed hline within table enviroment

% Used to make flowchart (ran out of blocks in lucidchart)
\usetikzlibrary{shapes.geometric, arrows}

%%%%%%%%%%%%%%%%%%%%%%%%%%%%%%%%%%%%%%%%%%%
%% IF YOU NEED A PACKAGE, PUT ABOVE THIS %%
%%%%%%%%%%%%%%%%%%%%%%%%%%%%%%%%%%%%%%%%%%%


%%%%%%%%%%%%%%%%%%%%%%%%%%%%%%%%%%%%%%%%%%%%%%%%
% Bibliography
% http://en.wikibooks.org/wiki/LaTeX/Bibliography_Management
%%%%%%%%%%%%%%%%%%%%%%%%%%%%%%%%%%%%%%%%%%%%%%%%
%\usepackage[square,numbers]{natbib}
% Add the \citep{key} command which display a
% reference as [author, year]
%\usepackage[authoryear]{natbib}
%\setcitestyle{round,nonamebreak}
%\bibpunct{(}{)}{;}{a}{}{,}
% Appearance of the bibliography
%\bibliographystyle{IEEEtran}




\addbibresource{kilder.bib}



%%%%%%%%%%%%%%%%%%%%%%%%%%%%%%%%%
%% THIS MUST BE THE LAST THING %%
%%%%%%%%%%%%%%%%%%%%%%%%%%%%%%%%%
%\usepackage[hidelinks,breaklinks]{hyperref}
%\hypersetup{%
%	%pdfpagelabels=true,%
%	plainpages=false,%
%	pdfauthor={gruppe 12,
%                3. semester,
%                AU,
%                Århus,
%                Denmark},%
%	pdftitle={Drink master},%
%	pdfsubject={Drink master},%
%	bookmarksnumbered=true,%
%	colorlinks=false,%
%	pdfstartview=FitH,%
%	pdfduplex=DuplexFlipLongEdge,
%	pdfkeywords={gruppe 12,
%                Århus,
%                AU,
%              },
%	breaklinks
%}

\usepackage{minted}
\definecolor{LightGray}{gray}{0.9}

\usepackage{memhfixc} %% Include this package after hyperref when using memoir