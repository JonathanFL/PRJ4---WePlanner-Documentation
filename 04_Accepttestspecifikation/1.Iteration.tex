\subsubsection{Accepttest for første iteration}
\valdemar{Skal overholde standarder fra Blackboard}
\noindent I den første iteration, er der valg at implementere 2 user stories fra bruger operationerne. Disse er valgt på baggrund af at der i første omgang skal laves et fundament for applikationen, for at gøre det nemmere implementere nogle af de andre user stories. Ud fra disse user stories er der lavet en acceptest beskrivelse med brug af Gherkin\cite{gherkin}.\\
Gherkin er et sprog designet til at beskrive adfærd af forskellige bruger scenarier og kan blive brugt til user stories, uden at inkludere logiske detaljer. Gherkin har en bestemt syntaks, hvilket gør at bruget af Gherkin, kan gøre produkt-specifikationer letlæselig for andre. Syntaksen består af at bruge keywords, som definerer hvad forskellige steps i en user story er. Projektgruppens user stories er skrevet på dansk, hvilket gør at accepttestspecifikationer også er skrevet på dansk. Danske karakterer giver lidt problemer, hvilket gør at der i stedet for \textit{æ, ø} og \textit{å} bliver brugt \textit{ae, o} og \textit{aa} \newline 

\noindent \textbf{\underline{Opret bruger}}\newline \textbf{Som} bruger\newline \textbf{Vil jeg} gerne kunne oprette mig som bruger til webappliaktionen med email og adgangskode\newline \textbf{For at} kunne logge ind til hjemmesiden. 

\begin{lstlisting}[language=Gherkin]
Egenskab: Opret bruger
    
    Baggrund:
        Givet at personen vil oprette en bruger paa webapplikationen, har en gyldig email
        
	Scenarie: Personen vil oprette en bruger
		Naar personen vil oprette en bruger til webapplikationen
		Saa navigerer personen til knappen "Opret en ny bruger" paa hjemmesiden
		Saa trykker personen paa knappen "Opret en ny bruger"
		Og bruger navigeres til siden hvor der kan oprettes en bruger
		Saa navigerer personen til tekst felterne for at indtaste email & adgangskode
		Saa indtaster personen email & adgangskode
		Og trykker opret bruger
		Saa bliver der oprettet en bruger til personen for webapplikationen, med de registrerede oplysninger
\end{lstlisting}

\noindent \textbf{\underline{Log ind}}\newline \textbf{Som} bruger\newline \textbf{Vil jeg} gerne kunne logge ind med email og adgangskode,\newline \textbf{For at} kunne få adgang til hjemmesiden og benytte mig af webapplikationen.

\begin{lstlisting}[language=Gherkin]
Egenskab: Log ind

    Baggrund:
        Givet at personen har oprettet en bruger til webapplikationen
        
	Scenarie: Personen vil logge ind
		Naar personen vil logge ind til webapplikationen
		Saa navigerer personen til omraadet for log ind paa hjemmesiden
		Saa trykker personen paa tekstboksene for email og adgangskode
		Og indtaster brugeroplysninger 
		Saa navigerer personen til knappen "Log ind" paa hjemmesiden
		Og trykker paa knappen "Log ind"
		Saa bliver brugeren logget ind 
		Og bliver videresendt til webapplikationen 
\end{lstlisting}
