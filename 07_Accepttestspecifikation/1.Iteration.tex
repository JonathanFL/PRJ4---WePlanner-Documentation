\subsubsection{Accepttest for første iteration}
\valdemar{Skal overholde standarder fra Blackboard}
\noindent I den første iteration, er der valg at implementere 2 user stories fra bruger operationerne. Disse er valgt på baggrund af at der i første omgang skal laves et fundament for applikationen, for at gøre det nemmere implementere nogle af de andre user stories. Ud fra disse user stories er der lavet en acceptest beskrivelse med brug af Gherkin\cite{gherkin}.\\
Gherkin er et sprog designet til at beskrive adfærd af forskellige bruger scenarier og kan blive brugt til user stories, uden at inkludere logiske detaljer. Gherkin har en bestemt syntaks, hvilket gør at bruget af Gherkin, kan gøre produkt-specifikationer letlæselig for andre. Syntaksen består af at bruge keywords, som definerer hvad forskellige steps i en user story er. Projektgruppens user stories er skrevet på dansk, hvilket gør at accepttestspecifikationer også er skrevet på dansk. Danske karakterer giver lidt problemer, hvilket gør at der i stedet for \textit{æ, ø} og \textit{å} bliver brugt \textit{ae, o} og \textit{aa} \newline 

\subsubsection{Bruger operationer}
\noindent \textbf{\underline{Opret bruger}}\newline \textbf{Som} bruger\newline \textbf{Vil jeg} gerne kunne oprette mig som bruger til webappliaktionen med email og adgangskode\newline \textbf{For at} kunne logge ind til hjemmesiden. 

\begin{lstlisting}[language=Gherkin]
Egenskab: Opret bruger
    
    Baggrund:
        Givet at personen vil oprette en bruger paa webapplikationen, har en gyldig email
        
	Scenarie: Personen vil oprette en bruger
		Naar personen vil oprette en bruger til webapplikationen
		Saa navigerer personen til knappen "Opret en ny bruger" paa hjemmesiden
		Saa trykker personen paa knappen "Opret en ny bruger"
		Og bruger navigeres til siden hvor der kan oprettes en bruger
		Saa navigerer personen til tekst felterne for at indtaste email & adgangskode
		Saa indtaster personen email & adgangskode
		Og trykker opret bruger
		Saa bliver der oprettet en bruger til personen for webapplikationen, med de registrerede oplysninger
\end{lstlisting}

\noindent \textbf{\underline{Log ind}}\newline \textbf{Som} bruger\newline \textbf{Vil jeg} gerne kunne logge ind med email og adgangskode,\newline \textbf{For at} kunne få adgang til hjemmesiden og benytte mig af webapplikationen.

\begin{lstlisting}[language=Gherkin]
Egenskab: Log ind

    Baggrund:
        Givet at personen har oprettet en bruger til webapplikationen
        
	Scenarie: Personen vil logge ind
		Naar personen vil logge ind til webapplikationen
		Saa navigerer personen til omraadet for log ind paa hjemmesiden
		Saa trykker personen paa tekstboksene for email og adgangskode
		Og indtaster brugeroplysninger 
		Saa navigerer personen til knappen "Log ind" paa hjemmesiden
		Og trykker paa knappen "Log ind"
		Saa bliver brugeren logget ind 
		Og bliver videresendt til webapplikationen 
\end{lstlisting}

\noindent\textbf{\underline{Indstil profilbillede }}\newline \textbf{Som} bruger\newline \textbf{Vil jeg} gerne kunne ændre profilbillede,\newline \textbf{For at} webapplikationen bliver mere personlig.

\begin{lstlisting}[language=Gherkin]
Egenskab: Indstil profilbillede

    Baggrund:
        Givet at personen har oprettet en bruger til webapplikationen og er logget ind
        
    Scenarie: Personen vil aendre profilbillede
        Naar personen vil aendre profilbillede
        Saa navigererer og trykker personen sig til brugerindstillinger paa hjemmesiden
        Saa trykker personen paa profile picture
        Og trykker saa paa "Vaelg fil"
        Saa navigerer personen gennem stien og finder et billede
        Saa trykker personen paa Aabn
        Og bliver navigeret tilbage til websiden
        Saa trykker personen paa "Upload Picture"
        

\end{lstlisting}

\noindent\textbf{\underline{Indstil personligt tema}}\newline \textbf{Som} bruger\newline \textbf{Vil jeg} gerne kunne ændre personligt tema,\newline \textbf{For at} webapplikationen bliver mere personlig.

\begin{lstlisting}[language=Gherkin]
Egenskab: Indstil personligt tema

    Baggrund:
        Givat at personen har oprettet en bruger til webappliaktionen og er logget ind
        
    Scenarie: Personen vil aendre tema
        Naar personen vil aendre tema
        Saa navigeres der til dropdown menuen i topbaren, hvor der staar "Theme"
        Saa trykker personen paa det oensket tema

\end{lstlisting}


\noindent\textbf{\underline{Ændring af navn}}\newline \textbf{Som} bruger\newline \textbf{Vil jeg} gerne have mulighed for at ændre mit eget navn\newline \textbf{For at} jeg har mulighed for tilpasning af personligt data.

\begin{lstlisting}[language=Gherkin]
Egenskab: Aendring af navn

    Baggrund:
        Givat at personen har oprettet en bruger til webapplikationen og er logget ind
        
    Scenarie: Personen vil aendre navn
        Naar personen vil aendre navn
        Saa navigerer og trykker personen til brugerindstillinger i topbaren
        Og trykker saa paa "profile" undermenuen
        Saa navigeres personen til profil indstillingerne
        Saa trykker personen paa tekstboksen under "Nick Name"
        Og sletter det forhenvaerende navn.
        Saa indtaster personen det oensket navn
        Og trykker paa "Save"

\end{lstlisting}

\noindent\textbf{\underline{Ændring af password}}\newline \textbf{Som} bruger\newline \textbf{Vil jeg} gerne have mulighed for at ændre mit eget password\newline \textbf{For at} jeg har mulighed for at opretholde sikkerhed.

\begin{lstlisting}[language=Gherkin]
Egenskab: Aendring af password

    Baggrund:
        Givet at personen har oprettet en bruger til webapplikationen og er logget ind
        
    Scenarie: Personen vil aendre password
        Naar personen vil aendre navn
        Saa navigerer og trykker personen til brugerindstillinger i topbaren
        Og trykker saa paa "Password" undermenuen
        Saa navigeres personen til password indstillingerne
        Saa indtaster personen det gamle password i den foerste tekstboks
        Og indtaster det nye password i naeste tekstboks
        Og indtaster det nye password igen i den sidste tekstboks
        Saa trykker personen paa "Save"
        
\end{lstlisting}

\noindent\textbf{\underline{Ændring af e-mail}}\newline \textbf{Som} bruger\newline \textbf{Vil jeg} gerne have mulighed for at ændre min egen e-mail\newline \textbf{For at} jeg har mulighed for tilpasning af kontakt og sikkerhed.

\begin{lstlisting}[language=Gherkin]
Egenskab: Aendring af e-mail

    Baggrund:
        Givet at personen har oprettet en bruger til webapplikationen og er logget ind
        
    Scenarie: Personen vil aendre e-mail
        Naar personen vil aendre navn
        Saa navigerer og trykker personen til brugerindstillinger i topbaren
        Og trykker paa "Profile"
        Saa navigeres personen til profil indstilligerne
        Saa navigerer personen til tekstboksen til e-email
        Og sletter den tidligere e-mail
        Saa indtasker personen den nye oensket e-mail
        Og trykker paa "Save"
        

\end{lstlisting}


\noindent\textbf{\underline{Opret gruppe}}\newline \textbf{Som} bruger\newline \textbf{Vil jeg} gerne kunne oprette en gruppe\newline \textbf{For at} jeg sammen med mit fælleskab kan administrere og planlægge fælles ressourcer og planer mm.

\begin{lstlisting}[language=Gherkin]
Egenskab: Oprettelse af gruppe

    Baggrund:
        Givet at personen har oprettet en bruger til webapplikationen og er logget ind
        
    Scenarie: Personen vil oprette en gruppe
        Naar vil oprette en gruppe
        Saa navigerer og trykker personen paa "Create Group" i venstre navigationsbar
        Saa navigeres personen til en side for oprettelse af gruppe
        Saa navigerer og trykker personen paa tekstboksen for Group name
        Og indtaster det oensket navn
        Saa navigerer og trykker personen paa "Create"
        Og personen navigeres til gruppens dashboard

\end{lstlisting}
\subsubsection{Gruppe indstillinger}

\textbf{\underline{Valg af tema, baggrund og gruppenavn}}\newline \textbf{Som} owner\newline \textbf{Vil jeg} gerne kunne ændre temaet, baggrunden og gruppenavnet\newline \textbf{For at} den bliver mere personlig og tilpasser sig gruppen. \\\\

\noindent\textbf{\underline{Tilføjelse/fjernelse af brugere}}\newline \textbf{Som} administrator\newline \textbf{Vil jeg} gerne kunne tilføje eller fjerne medlemmer til gruppen,\newline \textbf{For at} det stemmer overens med den fysiske gruppes tilstand. 

\begin{lstlisting}[language=Gherkin]
Egenskab: Tilfoej bruger

    Baggrund:
        Givet at personen er administrator i gruppen
        Og at personen er inde paa gruppens side
        Og at den bruger som skal tilfoejes er oprettet i databasen
        Og at brugeren som skal tilfoejes ikke allerede er medlem af gruppen
        
    Scenarie: Personen vil tilfoeje et nyt medlem til gruppen
        Naar Personen vil tilfoeje en bruger til gruppen
        Saa navigerer personen til knappen "Edit group" paa gruppens side
        Saa trykker personen paa knappen "Edit group"
        Og personen navigeres til siden hvor alle gruppe indstillinger er
        Saa navigerer personen til knappen "Add new member"
        Saa trykker personen paa knappen "Add new member"
        Og personen navigeres til et pop-up vindue, hvor det nye medlems brugernavn skal skrives
        Saa skriver personen brugerens brugernavn i tekstfeltet
        Saa navigerer personen til knappen "Add"
        Saa trykker personen paa knappen "Add"
        Og brugeren bliver tilføjet til personens gruppe.

\end{lstlisting}

\begin{lstlisting}[language=Gherkin]
Egenskab: Fjern bruger

    Baggrund:
        Givet at personen er administrator i gruppen
        Og at personen er inde paa gruppens side
        Og at den bruger som skal fjernes er medlem af gruppen
        Og at den brugeren som skal fjernes ikke er owner af gruppen
        
    Scenarie: Personen vil fjerne et medlem fra gruppen
        Naar Personen vil fjerne en bruger fra gruppen
        Saa navigerer personen til knappen "Edit group" paa gruppens side
        Saa trykker personen paa knappen "Edit group"
        Og personen navigeres til siden hvor alle gruppe indstillinger er
        Saa navigerer personen til knappen "Delete member" som er ud for brugeren navn.
        Saa trykker personen paa knappen "Delete member"
        Og personen navigeres til et pop-up vindue, hvor valget skal valideres
        Saa navigerer personen til knappen "Yes"
        Saa trykker personen paa knappen "Yes"
        Og brugeren bliver fjernet fra gruppen.

\end{lstlisting}

\textbf{\underline{Tilføje/fjerne administrator rettigheder}}\newline \textbf{Som} owner\newline \textbf{Vil jeg} gerne kunne tilføje eller fjerne administrator rettigheder til gruppens medlemmer,\newline \textbf{For at} jeg kan kontrollere hvem der har adgang til administrator-funktioner. 

\begin{lstlisting}[language=Gherkin]
Egenskab: Tilfoej administrator retiigheder til en bruger

    Baggrund:
        Givet at personen er administrator i gruppen
        Og at personen er inde paa gruppens side
        Og at brugeren som skal have administrator rettigheder er medlem af gruppen
        Og at brugeren som skal have administrator rettigheder ikke allerede har det
        
        
    Scenarie: Personen vil tilfoeje administrator rettigheder til et medlem af gruppen
        Naar Personen vil tilfoeje administrator rettigheder til en bruger i gruppen
        Saa navigerer personen til knappen "Edit group" paa gruppens side
        Saa trykker personen paa knappen "Edit group"
        Og personen navigeres til siden hvor alle gruppe indstillinger er
        Saa navigerer personen til checkboksen "Admin" som er ud for brugeren navn.
        Saa trykker personen paa checkboksen "Admin"
        Saa navigerer personen til knappen "Save"
        Saa trykker personen paa knappen "Save"
        Og brugeren faar tildelt administrator rettigheder til gruppen.

\end{lstlisting}

\begin{lstlisting}[language=Gherkin]
Egenskab: Fjerne administrator retiigheder til en bruger

    Baggrund:
        Givet at personen er administrator i gruppen
        Og at personen er inde paa gruppens side
        Og at brugeren som skal have fjernet administrator rettigheder er medlem af gruppen
        Og at brugeren som skal have fjernet administrator rettigheder har dem i forvejen
        
        
    Scenarie: Personen vil fjerne administrator rettigheder fra et medlem af gruppen
        Naar Personen vil fjerne administrator rettigheder fra en bruger i gruppen
        Saa navigerer personen til knappen "Edit group" paa gruppens side
        Saa trykker personen paa knappen "Edit group"
        Og personen navigeres til siden hvor alle gruppe indstillinger er
        Saa navigerer personen til checkboksen "Admin" som er ud for brugerens navn.
        Saa trykker personen paa checkboksen "Admin"
        Saa navigerer personen til knappen "Save"
        Saa trykker personen paa knappen "Save"
        Og brugeren får fjernet administrator rettighederne til gruppen.

\end{lstlisting}

\textbf{\underline{Mulighed for at forlade gruppe}}\newline \textbf{Som} bruger\newline \textbf{Vil jeg} gerne have mulighed for at forlade en gruppe,\newline \textbf{For at} jeg kan forlade gruppen hvis jeg ikke ønsker at være med i den længere.  
\begin{lstlisting}[language=Gherkin]
Egenskab: Forlade en gruppe

    Baggrund:
        Givet at personen er medlem af gruppen
        Og at personen er inde paa gruppens side
        
    Scenarie: Personen vil forlade en gruppe
        Naar Personen vil forlade en gruppe
        Saa navigerer personen til knappen "Leave group" paa gruppens side
        Saa trykker personen paa knappen "Leave group"
        Og personen navigeres til et pop-up vindue, hvor valget skal valideres
        Saa navigerer personen til knappen "Yes"
        Saa trykker personen paa knappen "Yes"
        Og personen har forladet gruppen.

\end{lstlisting}

\textbf{\underline{Mulighed for at slette gruppe}}\newline \textbf{Som} Owner\newline \textbf{Vil jeg} gerne have mulighed for at slette en gruppe\newline \textbf{For at} jeg kan slette gruppen hvis den ikke bliver brugt længere.

\begin{lstlisting}[language=Gherkin]
Egenskab: Slet gruppe

    Baggrund:
        Givet at personen er owner i gruppen
        Og at personen er inde paa gruppens side
        
    Scenarie: Personen vil slette en gruppe
        Naar Personen vil slette en gruppe
        Saa navigerer personen til knappen "Edit group" paa gruppens side
        Saa trykker personen paa knappen "Edit group"
        Og personen navigeres til siden hvor alle gruppe indstillinger er
        Saa navigerer personen til knappen "Delete group"
        Saa trykker personen paa knappen "Delete group"
        Og personen navigeres til et pop-up vindue, hvor valget skal valideres
        Saa navigerer personen til knappen "Delete group"
        Saa trykker personen paa knappen "Delete group"
        Og personen har slettet gruppen.

\end{lstlisting}

\textbf{\underline{Tilføje/fravælge notifikationer}}\newline \textbf{Som} bruger\newline \textbf{Vil jeg} gerne have mulighed for at tilføje/fravælge notifikationer i appen\newline \textbf{For at} jeg kan bestemme hvilke typer af notifikationer jeg gerne vil se.  \\\\

\textbf{\underline{Mulighed for at tilføje kælenavne}}\newline \textbf{\underline{til medlemmer}}\newline \textbf{Som} administrator\newline \textbf{Vil jeg} gerne have mulighed for at tilføje kælenavne til en eller flere af gruppens medlemmer\newline \textbf{For at} gruppen personliggøres.

\begin{lstlisting}[language=Gherkin]
Egenskab: Tilfoej kaelenavn til en bruger i en gruppe

    Baggrund:
        Givet at personen er administrator i gruppen
        Og at personen er inde paa gruppens side
        Og at brugeren som skal have et kaelenavn er medlem af gruppen
        
    Scenarie: Personen tilfoeje et kaelenavn til et medlem i gruppen
        Naar Personen vil tilfoeje et kaelenavn til et en bruger i gruppen
        Saa navigerer personen til knappen "Edit group" paa gruppens side
        Saa trykker personen paa knappen "Edit group"
        Og personen navigeres til siden hvor alle gruppe indstillinger er
        Saa navigerer personen til tekstfeltet under Nickname som er ud for brugerens navn
        Saa skriver personen et kaelenavn i tekstfeltet
        Saa navigerer personen til knappen "Save"
        Saa trykker personen paa knappen "Save"
        Og brugeren får tildelt et kaelenavn til gruppen.

\end{lstlisting}

\textbf{\underline{Tilføj widget}}\newline \textbf{Som} bruger\newline \textbf{Vil jeg} gerne have mulighed for at tilføje widgets på gruppens side\newline \textbf{For at} jeg kan administrere indholdet på gruppens side og den kan tilpasses til gruppens behov.



\textbf{\underline{Fjern widget}}\newline \textbf{Som} administrator\newline \textbf{Vil jeg} gerne have mulighed for at fjerne widgets på gruppens side\newline \textbf{For at} jeg kan administrere indholdet på gruppens side så den kan tilpasses til gruppens behov.  \\\\

\textbf{\underline{Mulighed for at videregive owner rettigheder}}\newline \textbf{Som} owner\newline \textbf{Vil jeg} gerne have mulighed for at videre give owner rettighederne\newline \textbf{For at} jeg kan give en anden mulighed for at styre gruppen.  \\\\
\subsubsection{Kalender}

\noindent \textbf{\underline{Tilføj/fjern begivenheder}}\newline \textbf{Som} bruger\newline \textbf{Vil jeg} gerne kunne tilføje/fjerne begivenhder
\newline \textbf{For at} kunne koordinere med bofællesskabet. 

\begin{lstlisting}[language=Gherkin]
Egenskab: Tilfoej begivenhed

    Baggrund:
        Givet at personen har oprettet en bruger og er logget ind
        Og at der findes en gruppe for brugeren
        Og at kalenderen er oprettet i en gruppe som en widget
        Og at brugeren er medlem af den samme gruppe som kalenderen findes i
        Og at personen er navigeret til den selvsamme kalender
        
    Scenarie: Personen vil tilfoeje en begivenhed
        Naar personen vil tilfoeje en begivenhed i kalenderen
        Saa navigerer personen til knappen "Insert event" og trykker på knappen
        Og brugeren navigeres til siden hvor der kan indsaettes en begivenhed
        Saa navigerer personen til tekst felterne for at indtaste information omkring begivnheden som skal indsaettes
        Saa indtaster personen oplysningerne
        Og navigerer til knappen "Save Event" og trykker på den
        Saa bliver der oprettet en begivenhed i kalenderen, med de registrerede oplysninger
        Og brugeren bliver navigeret til kalenderen igen

\end{lstlisting}

\begin{lstlisting}[language=Gherkin]
Egenskab: Fjern begivenheder

    Baggrund:
        Givet at personen har oprettet en bruger og er logget ind
        Og at der findes en gruppe for brugeren
        Og at kalenderen er oprettet i en gruppe som en widget
        Og at brugeren er medlem af den samme gruppe som kalenderen findes i
        Og at personen er navigeret til den selvsamme kalender
        Og at der er mindst en begivenhed i kalenderen
        
    Scenarie: Personen vil fjerne en begivenhed
        Naar personen vil fjerne en begivenhed fra kalenderen
        Saa navigerer personen til det event som vedkommenede oensker at fjerne, og trykker på eventet
        Og personen praesenteres for et pop up vindue til begivenheden
        Saa navigerer personen til knappen "Remove" for at fjerne begivenheden
        Og trykker på knappen
        Saa bliver personen praesenteret for et pop up vindue, hvor brugeren kan konfirmere aktionen
        Og personen trykker på "ok"
        Saa bliver begivenheden fjernet fra kalenderen

\end{lstlisting}

\noindent \textbf{\underline{Skift mode}}\newline \textbf{Som} bruger\newline \textbf{Vil jeg} kunne vælge mellem fælles/personlig visning.
\newline \textbf{For at} kunne overskue hhv. fælles og personlige planer.

\begin{lstlisting}[language=Gherkin]
Egenskab: Skift 

    Baggrund:
        Givet at personen har oprettet en bruger og er logget ind
        Og at der findes en gruppe for brugeren
        Og at kalenderen er oprettet i en gruppe som en widget
        Og at brugeren er medlem af den samme gruppe som kalenderen findes i
        Og at personen er navigeret til den selvsamme kalender
        
    Scenarie: xxx
        Naar xxx

\end{lstlisting}

\noindent \textbf{\underline{Valg af filter}}\newline \textbf{Som} bruger\newline \textbf{Vil jeg} gerne have mulighed for at filtrere begivenheder i kalenderen.
\newline \textbf{For at} kunne se brugerdefinerede type af begivenheder.

\begin{lstlisting}[language=Gherkin]
Egenskab: Valg af filter

    Baggrund:
        Givet at personen har oprettet en bruger og er logget ind
        Og at der findes en gruppe for brugeren
        Og at kalenderen er oprettet i en gruppe som en widget
        Og at brugeren er medlem af den samme gruppe som kalenderen findes i
        Og at personen er navigeret til den selvsamme kalender 
        
    Scenarie: Personen vil filtrere begivenheder fra, saaledes at kun bestemte begivenheder     vises 
        Naar personen vil vaelge et filter
        Saa navigerer personen til knappen dropdown menuen ved knappen "Filter", og trykker
        Saa praesenteres personen for en liste af muligheder for at filtrere i begivenhederne
        Og personen vaelger og trykker paa en af mulighederne
        Saa navigerer personen til knappen "Filter" og trykker på knappen
        Saa viser kalenderen begivenhederne med begivenhedstype som matcher filteret

\end{lstlisting}

\noindent \textbf{\underline{Valg af visning}}\newline \textbf{Som} bruger\newline \textbf{Vil jeg} gerne kunne vælge mellem visning af måned/uge
\newline \textbf{For at} få overblik af kalenderen.

\begin{lstlisting}[language=Gherkin]
Egenskab: xxx

    Baggrund:
        xxx
        
    Scenarie: xxx
        Naar xxx

\end{lstlisting}




\begin{lstlisting}[language=Gherkin]
Egenskab: xxx

    Baggrund:
        xxx
        
    Scenarie: xxx
        Naar xxx

\end{lstlisting}



\begin{lstlisting}[language=Gherkin]
Egenskab: xxx

    Baggrund:
        xxx
        
    Scenarie: xxx
        Naar xxx

\end{lstlisting}