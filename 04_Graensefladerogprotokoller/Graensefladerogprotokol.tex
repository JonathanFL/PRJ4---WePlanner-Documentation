\chapter{Grænseflader og protokoller}

\section{Grænseflader}
Da det fra projektets side er besluttet at bruge MVC Frameworket er der en meget fast standard for hvordan kommunikationen på WePlanner webapplikationen skal foregå. Webapplikationen tager nemlig brug af HTTP Hypertext Transfer Protocol, som er en protokol der tager udgangspunkt i client-server princippet om at en client requester fra en server, hvorefter serveren sender en respons tilbage. HTTP er bygget op af en anden protokol, TCP, som basalt bliver brugt til at sende data mellem client og server. HTTP bruger altså TCP på et niveau hvor HTTP har defineret en protokol, sådan at der kan sendes data på ordenlig vis. I den protokol er der nogle bestemte handlinger, som kan bruges og skal overholdes, for at tage brug af HTTP. Det gælder i den sammenhænge om at det er clienten der sender en forespørgsel til serveren, hvor den bruger den rigtige syntaks. De mest anvendte forespørgsler er GET og POST\cite{httpWiki}. Disse 2 forespørgsler demonstrere meget godt hvordan et TCP forhold er, hvor en client forespørger om at få noget data(GET) hvor serveren derefter sender dataen tilbage. På den anden side kan det også ses banalt som et read/write forhold, hvor GET er read og POST er write. POST er nemlig for den forespørgsel, hvor clienten gerne vil ændre noget data i fx en database.

\subsection{HTTPS}

Som en tilføjelse til HTTP er det også muligt at have mere sikkerhed på sin hjemmeside, ved at tage brug af protokollen HTTP\textbf{S}. I HTTPS står S'et for Secure, hvilket betyder at HTTPS beskytter ens hjemmeside. HTTPS er vigtig at brug hvis en hjemmeside skal håndtere informationer som CPR nummer eller dankort oplysninger. Da WePlanner ikke har planner om at tage brug af at store andet end mail, navn og password, har det ikke været set nødvendigt at tage brug af HTTPS.\newline Da projektet bliver udviklet i ASP.NET Core med MVC Frameworket, og der i WePlanner er brug for at have individuelle brugere, tages der brug af ASP.NET Core's egen Individual User Accounts\cite{aspDotNetUserAccounts}. Denne funktionalitet fra ASP.NET sørger for at passwords er krypterede, så der ikke ligger helt tilgængelige passwords i WePlanners database.


\subsection{Pilotprojekter og tilegnelse af viden}
For at danne et indtryk af MVC og samtidig opnå viden om webudvikling, blev der som forarbejde lavet projekter i MVC, der skulle vise hvordan strukturen var og hvilke muligheder MVC gav, for at genneføre WePlanner. Det var bl.a. sagt fra gruppens vejleder, at ASP.NET MVC havde mulighed for at have individuelle brugere, og allerede login og registrering af brugere sat op. Dette var en væsentlig faktor om hvorfor MVC blev valgt, da WePlanner vil komme til at være meget afhængige af mange forskellige brugere og gruppeinddelinger. Desuden hjalp de første pilotprojekter også med at danne et godt indblik i hvordan MVC håndterede design princippet Separation of Concern\cite{separationOfConcern}, ved at inddele hele applikationen i Models, Views og Controllers. Controllers og Models forhold til views var som udgangspunkt 1 til mange, som gik ud på at der kunne være én controller, der passede én model til forskellige views.

% \valdemar{Opret 1 pilot-projekt i MVC}
% HTTP, HTTPS - version 0.9, 1.0 eller 1.1, TCP. Web devel med MVC. Hvor ligger alt processering af data? Gør man det sådan at det kun er front end der bliver compiled når bruger går på hjemmeside og så når man trykker på noget, kalder front-end noget back-end på en server? Hvilken slags kommunikation er det? HTTP post, get, put? Database kommunikation - hvordan fungerer det?
% The goal of MVC and related patterns is to separate presentation from domain business logic. 